\documentclass[t]{beamer}
%\documentclass[finnish,english,handout]{beamer}

% Uncomment if want to show notes
% \setbeameroption{show notes}

\mode<presentation>
{
  \usetheme{Copenhagen}
  % oder ...

  %\setbeamercovered{transparent}
  % oder auch nicht
}


\usepackage[latin1]{inputenc}
\usepackage{times}
\usepackage{url}
\urlstyle{same}
\usepackage{hyperref}
% \usepackage{enumerate}
% \usepackage{parskip}
% \usepackage{enumitem}% http://ctan.org/pkg/enumitem

\hypersetup{%
  bookmarksopen=true,
  bookmarksnumbered=true,
  pdftitle={Stan},
  pdfsubject={Bayesian data analysis},
  pdfauthor={Aki Vehtari},
  pdfkeywords={},
  pdfstartview={FitH -32768},
  colorlinks=true,
  linkcolor=navyblue,
  citecolor=navyblue,
  filecolor=navyblue,
  urlcolor=navyblue
}


% \definecolor{hutblue}{rgb}{0,0.2549,0.6784}
% \definecolor{midnightblue}{rgb}{0.0977,0.0977,0.4375}
% \definecolor{hutsilver}{rgb}{0.4863,0.4784,0.4784}
% \definecolor{lightgray}{rgb}{0.95,0.95,0.95}
% \definecolor{section}{rgb}{0,0.2549,0.6784}
% \definecolor{list1}{rgb}{0,0.2549,0.6784}
\definecolor{forestgreen}{rgb}{0.1333,0.5451,0.1333}
\definecolor{navyblue}{rgb}{0,0,0.5}
\renewcommand{\emph}[1]{\textcolor{navyblue}{#1}}

%\graphicspath{{./figs/}}

\pdfinfo{
  /Title      (Bayesian data analysis)
  /Author     (Aki Vehtari) %
  /Keywords   (Bayesian probability theory, Bayesian inference, Bayesian data analysis)
}


\parindent=0pt
\parskip=8pt
\tolerance=9000
\abovedisplayshortskip=0pt

\setbeamertemplate{navigation symbols}{}
\setbeamertemplate{headline}[default]{}
\setbeamertemplate{headline}[text line]{\insertsection}
\setbeamertemplate{footline}[frame number]


\title[]{Bayesian data analysis}
\subtitle{}

\author{Aki Vehtari}

\institute[Aalto]{}

\begin{document}

\begin{frame}

  {\Large\color{navyblue} Project work}
  
  \begin{itemize}
  \item Choose a data set and make all the steps of Bayesian data
    analysis workflow listed below
  \item Project outcome is a Python or R notebook similar to notebooks
    in (many of these notebooks don't have all the required parts)
    \begin{itemize}
    \item BDA R demos \url{https://github.com/avehtari/BDA_R_demos/tree/master/demos_rstan}
    \item BDA Python demos \url{https://github.com/avehtari/BDA_py_demos/tree/master/demos_pystan}
    \item Stan case studies \url{http://mc-stan.org/users/documentation/case-studies.html}
    \item StanCon case studies \url{http://mc-stan.org/users/documentation/case-studies.html}
      (some of these notebooks are for a bigger projects, but reflect still the basic idea of a notebook presentation)
    \end{itemize}
  \item The submitted notebooks need to illustrate the knowledge of the
    Bayesian workflow.
  \end{itemize}
\end{frame}

\begin{frame}
  
  {\Large\color{navyblue} Project work}
  
  \begin{itemize}
  \item The notebooks have to include
    \begin{itemize}
    \item Description of the data, and the analysis problem
    \item Description of the model
    \item Description of the prior choices
    \item Stan code
    \item How Stan model is run
    \item Convergence diagnostics (Rhat, divergences, neff)
    \item Posterior predictive checking
    \item Model comparison (e.g. with loo)
    \item Predictive performance assessment if applicable (e.g. classification
      accuracy)
    \item Potentially sensitivity analysis
    \item Discussion of problems, and potential improvements 
      \begin{itemize}
      \item It is possible that your model or inference is not perfect, but a better model would require substantial work. Then it's ok that you report the problems found (using the various diagnostics discussed in the course) and describe possible improvements.
      \end{itemize}
    \end{itemize}
  \end{itemize}
\end{frame}

\begin{frame}
  
  {\Large\color{navyblue} Project work}
  
  \begin{itemize}
  \item You can re-use of code and text from existing case studies
    \begin{itemize}
    \item Just report what did you re-use
    \item Acknowledge the original authors
    \item Include the original copyright licence
      \begin{itemize}
      \item CC-BY or CC-BY-NC is common for text
        \url{https://creativecommons.org/licenses/}
      \item BSD-3 is common for code
        \url{https://opensource.org/licenses/BSD-3-Clause}
    \end{itemize}
    \end{itemize}
  \item You can use BRMS to create Stan code, but don�t limit yourself
    to BRMS models if changes would make a better model
  \end{itemize}
\end{frame}

\begin{frame}
  
  {\Large\color{navyblue} Oral presentation}
  
  \begin{itemize}
  \item During evaluation week 50
  \item Max 8min oral presentation with slides, discussion and evaluation
  \end{itemize}
\end{frame}

\begin{frame}
  
  {\Large\color{navyblue} Some special topics}

  \begin{itemize}
  \item Update Python demos to use ArviZ
  \item Dynamic HMC demo in R or Python
  \end{itemize}
  
\end{frame}

\begin{frame}
  
  {\Large\color{navyblue} Some ideas for data sets}
  
  \begin{itemize}
  \item Laptop multitasking hinders classroom learning for both users
    and nearby peers
    \url{http://www.sciencedirect.com/science/article/pii/S0360131512002254}
  \item Arctic sea ice shrinking \url{https://www.nytimes.com/interactive/2017/09/22/climate/arctic-sea-ice-shrinking-trend-watch.html}
  \item Finnish weather statistics \url{https://en.ilmatieteenlaitos.fi/statistics-from-1961-onwards}
  \item R datasets \url{https://vincentarelbundock.github.io/Rdatasets/datasets.html}
  \item Vanderbilt Biostatistics \url{http://biostat.mc.vanderbilt.edu/wiki/Main/DataSets}
  \item Probaly better to *not* have a data set
    \begin{itemize}
    \item with number of observation in millions
    \item machine vision task
    \end{itemize}
  \end{itemize}
\end{frame}

\begin{frame}
  
  {\Large\color{navyblue} Schedule}


  \begin{itemize}
  \item Register project group and topic by 5th November
  \item During the week starting 5th November, start working on the
    project and if necessary talk with TAs (now new assignment on that
    week)
  \item Deadline end of week 49, 9 December
  \item Oral presentations during the evaluation week (week 50)
  \end{itemize}
  
\end{frame}

\end{document}

%%% Local Variables: 
%%% TeX-PDF-mode: t
%%% TeX-master: t
%%% End: 
