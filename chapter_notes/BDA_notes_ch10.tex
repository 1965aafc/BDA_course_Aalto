\documentclass[a4paper,11pt,english]{article}

\usepackage{babel}
\usepackage[latin1]{inputenc}
\usepackage[T1]{fontenc}
\usepackage{times}
\usepackage{amsmath}
\usepackage{microtype}
\usepackage{url}
\urlstyle{same}

\usepackage[bookmarks=false]{hyperref}
\hypersetup{%
  bookmarksopen=true,
  bookmarksnumbered=true,
  pdftitle={Bayesian data analysis},
  pdfsubject={Comments},
  pdfauthor={Aki Vehtari},
  pdfkeywords={Bayesian probability theory, Bayesian inference, Bayesian data analysis},
  pdfstartview={FitH -32768},
  colorlinks=true,
  linkcolor=black,
  citecolor=black,
  filecolor=black,
  urlcolor=black
}


% if not draft, smaller printable area makes the paper more readable
\topmargin -4mm
\oddsidemargin 0mm
\textheight 225mm
\textwidth 160mm

%\parskip=\baselineskip

\DeclareMathOperator{\E}{E}
\DeclareMathOperator{\Var}{Var}
\DeclareMathOperator{\var}{var}
\DeclareMathOperator{\Sd}{Sd}
\DeclareMathOperator{\sd}{sd}
\DeclareMathOperator{\Bin}{Bin}
\DeclareMathOperator{\Beta}{Beta}
\DeclareMathOperator{\Invchi2}{Inv-\chi^2}
\DeclareMathOperator{\NInvchi2}{N-Inv-\chi^2}
\DeclareMathOperator{\logit}{logit}
\DeclareMathOperator{\N}{N}
\DeclareMathOperator{\U}{U}
\DeclareMathOperator{\tr}{tr}
%\DeclareMathOperator{\Pr}{Pr}
\DeclareMathOperator{\trace}{trace}

\pagestyle{empty}

\begin{document}
\thispagestyle{empty}

\section*{Bayesian data analysis -- reading instructions ch 10} 
\smallskip
{\bf Aki Vehtari}
\smallskip

\subsection*{Chapter 10}

Outline of the chapter 10
\begin{list}{$\bullet$}{\parsep=0pt\itemsep=2pt}
\item 10.1 Numerical integration (overview)
\item 10.2 Distributional approximations (overview, more in Chapter 4 and 13)
\item 10.3 Direct simulation and rejection sampling (overview)
\item 10.4 Importance sampling (used in PSIS-LOO discussed later)
\item 10.5 How many simulation draws are needed? (Important! Ex 10.1 and 10.2)
\item 10.6 Software (can be skipped)
\item 10.7 Debugging (can be skipped)
\end{list}

Sections 10.1-10.4 give overview of different computational
methods. Some of then have been already used in the book.

Section 10.5 is very important and related to the exercises.

Demos
\begin{list}{$\bullet$}{\parsep=0pt\itemsep=2pt}
\item demo10\_1: Rejection sampling
\item demo10\_2: Importance sampling
\end{list}

Find all the terms and symbols listed below. When reading the chapter,
write down questions related to things unclear for you or things you
think might be unclear for others. 
\begin{list}{$\bullet$}{\parsep=0pt\itemsep=2pt}
\item unnormalized density
\item target distribution
\item log density
\item overflow and underflow
\item numerical integration
\item quadrature
\item simulation methods
\item Monte Carlo
\item stochastic methods
\item deterministic methods
\item distributional approximations
\item crude estimation
\item direct simulation
\item grid sampling
\item rejection sampling
\item importance sampling
\item importance ratios/weights
\end{list}

 \subsection*{Quadrature}

 Sometimes `quadrature' is used to refer generically to any numerical
 integration method (including Monte Carlo), sometimes it is used to
 refer just to deterministic numerical integration methods.

 \subsection*{Rejection sampling}

 Rejection sampling is mostly used as a part of fast methods for
 univariate sampling. For example, sampling from the normal
 distribution is often made using Ziggurat method, which uses a
 proposal distribution resembling stairs.

 \subsection*{Importance sampling}

 Popularity of importance sampling is increasing. It is used, for
 example, as part of other methods as particle filters and pseudo
 marginal likelihood approaches, and to improve distributional
 approximations.

 Importance sampling is useful in importance sampling leave-one-out
 cross-validation. Cross-validation is discussed in Chapter 7 and
 importance sampling leave-one-out cross-validation is discussed in
 the article
 \begin{itemize}
 \item Aki Vehtari, Andrew Gelman and Jonah Gabry (2016). Practical
   Bayesian model evaluation using leave-one-out cross-validation and
   WAIC. In Statistics and Computing, 27(5):1413--1432. arXiv preprint
   arXiv:1507.04544 \url{<http://arxiv.org/abs/1507.04544>}
 \end{itemize}

 \subsection*{Buffon's needles}

 Computer simulation of Buffon's needle dropping method for estimating
 the value of $\pi$ \url{http://www.metablake.com/pi.swf}. 

 
\end{document}

%%% Local Variables: 
%%% TeX-PDF-mode: t
%%% TeX-master: t
%%% End: 
