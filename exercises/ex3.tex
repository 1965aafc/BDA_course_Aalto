\documentclass[a4paper,11pt]{article}

\usepackage[pdftex]{graphicx}
%\usepackage{babel}
\usepackage[utf8]{inputenc}
\usepackage[T1]{fontenc}
%\usepackage[T1,mtbold,lucidacal,mtplusscr,subscriptcorrection]{mathtime}
\usepackage{times}
\usepackage{amsmath}
\usepackage{url}
\urlstyle{same}
\usepackage{enumerate}
\usepackage{parskip}
\usepackage{hyperref}
\usepackage{microtype}


%\usepackage[dvips,bookmarks=false]{hyperref}
% \hypersetup{%
%   bookmarksopen=true,
%   bookmarksnumbered=true,
%   pdftitle={S-114.2601 Bayesilaisen mallintamisen perusteet},
%   pdfsubject={Kommentteja},
%   pdfauthor={Aki Vehtari},
%   pdfkeywords={bayesilainen todennäköisyysteoria, bayesilainen
%     päättely, bayesilaiset mallit, mallien analysointi,
%     laskennalliset menetelmät, Markov-ketju Monte Carlo},
%   pdfstartview={FitH -32768}
% }


% if not draft, smaller printable area makes the paper more readable
\topmargin -4mm
\oddsidemargin 0mm
\textheight 225mm
\textwidth 150mm

%\parskip=\baselineskip

\DeclareMathOperator{\E}{E}
\DeclareMathOperator{\Var}{Var}
\DeclareMathOperator{\var}{var}
\DeclareMathOperator{\Sd}{Sd}
\DeclareMathOperator{\sd}{sd}
\DeclareMathOperator{\Bin}{Bin}
\DeclareMathOperator{\Beta}{Beta}
\DeclareMathOperator{\Poisson}{Poisson}
\DeclareMathOperator{\betacdf}{betacdf}
\DeclareMathOperator{\Invchi2}{Inv-\chi^2}
\DeclareMathOperator{\logit}{logit}
\DeclareMathOperator{\N}{N}
\DeclareMathOperator{\U}{U}
\DeclareMathOperator{\tr}{tr}
\DeclareMathOperator{\trace}{trace}

\pagestyle{empty}

\begin{document}
\thispagestyle{empty}

\section*{Bayesian data analysis -- exercise 3}

This exercise is related to Chapters 2 and 3.

The maximum amount of points from this assignment is 9. In addition to the correctness of the answers, the overall quality and clearness of the report is evaluated.

Report all results to a single, {\bf anonymous} *.pdf -file and return it to \href{peergrade.io}{peergrade.io}. Include also source code to the report (either as an attachment or as a part of the answer). By anonymity it is meant that the report should not contain your name or student number.

\vspace{1cm}

You may find Frank Harrell's recommendations how to state results in
two group comparisons helpful
\url{http://www.fharrell.com/2017/10/bayesian-vs-frequentist-statements.html}.

\subsection*{1. Inference for normal mean and deviation (3 points)}

A factory has a production line for manufacturing car windshields. 
A sample of windshields has been taken for testing hardness. The
observed hardness values $\mathbf{y}_1$ can be found in file 
{\tt windshieldy1.txt}.

We may assume that the observations follow a normal distribution with
an unknown standard deviation $\sigma$. We wish to obtain information
about the unknown average hardness $\mu$. Formulate the Bayesian model
with uninformative or weakly informative prior and answer the
following questions below. Here it is not necessary to derive the posterior distribution as it has already been done in the book. 
\begin{enumerate}[a)]
\item What can you say about the unknown $\mu$? Summarize your results
  using Bayesian point and interval estimates and plot the density. 
\item What can you say about the hardness of the next windshield
  coming from the production line before actually measuring the
  hardness? Summarize your results using Bayesian point and interval estimates and plot the density.
\end{enumerate}
Hint: With a conjugate prior a closed form posterior is Student's $t$
form (see equations in the book). 
Python users can use {\tt scipy.stats.t} module. It has both {\tt pdf} and {\tt cdf} functions, where you can also set the mean with the  loc argument and variance with scale argument. Note that these functions are for standardized
Student-$t$ with mean 0 and scale 1, so you need to do the scaling and
translation yourself.
R users can use the {\tt dt} function after doing input normalisation. We have added a R file having {\tt dtnew} function which does that in the attachments. When you generate samples, use the standard {\tt dt} function and transform them back.


\subsection*{2. Inference for difference between proportions (3 points)}

An experiment was performed to estimate the effect of beta-blockers
on mortality of cardiac patients. A group of patients were randomly
assigned to treatment and control groups: out of 674 patients
receiving the control, 39 died, and out of 680 receiving the
treatment, 22 died. Assume that the outcomes are independent and
binomially distributed, with probabilities of death of $p_0$ and $p_1$
under the control and treatment, respectively. Set up a noninformative
or weakly informative prior distribution on $(p_0,p_1)$. 
\begin{enumerate}[a)]
\item Summarize the posterior distribution for the odds ratio, $(p_1/(1-p_1))/(p_0/(1-p_0))$. Compute the point and interval estimates and plot the histogram. 
\item Discuss the sensitivity of your inference
to your choice of prior density with a couple of sentences.
\end{enumerate}
Hint: With a conjugate prior a closed form posterior is Beta form for
each group separately (see equations in the book). You can use {\tt scipy.stats.beta.rvs} in Python
to sample from the posterior distrbutions of $p_0$ and $p_1$, and use
these samples and odds ratio equation to get samples from the
distribution of the odds ratio.
Python users can make use of commands like {\tt numpy.percentile} to get percentiles. For plotting, matplotlib package is recommended.

\subsection*{3. Inference for difference between normal means (3 points)}

Consider a case where the same factory has two production lines for
manufacturing car windshields. Independent samples from the two
production lines were tested for hardness. The hardness measurements
for the two samples $\mathbf{y}_1$ and $\mathbf{y}_2$ are given in the
files {\tt windshieldy1.txt} and {\tt windshieldy2.txt}.

We assume that the samples have unknown standard deviations $\sigma_1$
and $\sigma_2$. Use uninformative or weakly informative priors and
answer the following questions:
\begin{enumerate}[a)]
\item What can you say about $\mu_d = \mu_1 - \mu_2$? Summarize
  your results using Bayesian point and interval estimates and plot the histogram.     
\item Are the means the same? Explain your reasoning with a couple of sentences.
\end{enumerate}
Hint: With a conjugate prior a closed form posterior is Student's $t$
form for each group separately (see equations in the book). You can
use {\tt scipy.stats.t.rvs} function to sample from the posterior distrbutions of $\mu_1$ and
$\mu_2$ in Python, and use these samples to get samples from the distribution of
the difference $\mu_d = \mu_1 - \mu_2$. The equivalent function in R is the {\tt rt} function. Be careful to scale them and shift them according to their mean and variance values in R, as described above.

Python users can use {\tt matplotlib.pyplot.hist} and {\tt numpy.percentile} for visualisation of histogram and computing intervals respectively.


\end{document}

%%% Local Variables:
%%% mode: latex
%%% TeX-master: t
%%% End:
