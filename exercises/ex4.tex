\documentclass[11pt,a4paper,english]{article}
%\usepackage[finnish]{babel}
\usepackage[latin1]{inputenc}
\usepackage{hyperref}
\usepackage{listings}

\usepackage{xcolor}
\hypersetup{
    colorlinks,
    linkcolor={red!50!black},
    citecolor={blue!50!black},
    urlcolor={blue!80!black}
}

 % Normal latex T1-fonts
\usepackage[T1]{fontenc}
 % For the pdf-conversion. Remember -Ppdf for latex!
%\usepackage[T1,mtbold,lucidacal,mtplusscr,subscriptcorrection]{mathtime}
%\usepackage{times}

 % Figures
\usepackage{graphicx,fancybox,subfigure}

 % AMS-stuff
\usepackage{amsmath,amsfonts,amssymb}
\usepackage{amsbsy}

 % Misc
\usepackage{verbatim}
\usepackage{url}

% Page size
\addtolength{\hoffset}{-1cm}
\addtolength{\textwidth}{2cm}
\addtolength{\voffset}{-1cm}
\addtolength{\textheight}{2cm}

% Paragraph
\setlength{\parindent}{0pt}
\setlength{\parskip}{1ex plus 0.5ex minus 0.2ex}

% Horizontal line
\newcommand{\HRule}{\rule{\linewidth}{0.5mm}}

% No page numbers
\pagestyle{empty}

% New commands:
\DeclareMathOperator{\var}{var}
\DeclareMathOperator{\cov}{cov}
\newcommand{\E}{\mathbb{E}}
\newcommand{\Prob}{\mathbb{P}}


\usepackage{Sweave}
\begin{document}
\input{ex4-concordance}

\section*{Bayesian Data Analysis - Assignment 4}



\HRule


\subsubsection*{General information}

\begin{itemize}
\itemsep0em 
\item The recommended tool in this course is R (with the IDE R-Studio). You can download R \href{https://cran.r-project.org/}{\textbf{here}} and R-Studio \href{https://www.rstudio.com/products/rstudio/download/}{\textbf{here}}. There are tons of tutorials, videos and introductions to R and R-Studio online. You can find some initial hints \href{https://www.rstudio.com/online-learning/}{\textbf{here}}. 
\item  You can write the report with your preferred software, but the outline of the report should follow the instruction in the R markdown template that can be found \href{https://raw.githubusercontent.com/avehtari/BDA_course_Aalto/master/templates/assignment_template.rmd}{\textbf{here}}. 
\item  Report all results in a single, {\bf anonymous} *.pdf -file and return it to \href{peergrade.io}{\textbf{peergrade.io}}. 
\item Many of the exercises can be checked using the R package \texttt{markmyassignment}. Information on how to install and use the package can be found \href{https://cran.r-project.org/web/packages/markmyassignment/vignettes/markmyassignment.html}{\textbf{here}}.
\item The course has its own R package with data and functionality to simplify coding. To install the package just run the following:
\begin{enumerate}
\item \texttt{install.packages("devtools")}
\item \texttt{devtools::install\_github("avehtari/BDA\_course\_Aalto", \\ subdir = "rpackage")}
\end{enumerate}
\item Many of the exercises can be checked automatically using the R package \\ \texttt{markmyassignment}. Information on how to install and use the package can be found \href{https://cran.r-project.org/web/packages/markmyassignment/vignettes/markmyassignment.html}{\textbf{here}}.
\item Additional self study exercises and solutions for each chapter in BDA3 can be found \href{http://www.stat.columbia.edu/~gelman/book/solutions3.pdf}{\textbf{here}}.
\item If you have any suggestions or improvements to the course material, please feel free to create an issue or submit a pull request to the public repository!!
\end{itemize}

\newpage

\subsubsection*{Information on this assignment}

This exercise is related to Chapters 3 and 10. The maximum amount of points from this assignment is 6.


\textbf{Reading instructions:} Chapters 3 and 10 in BDA3, see reading instructions \href{https://github.com/avehtari/BDA_course_Aalto/blob/master/chapter_notes/BDA_notes_ch3.pdf}{\textbf{here}} and \href{https://github.com/avehtari/BDA_course_Aalto/blob/master/chapter_notes/BDA_notes_ch10.pdf}{\textbf{here}}

\textbf{Grading instructions:} The grading will be done in peergrade. All grading questions and evaluations for exercise 4 can be found \href{https://github.com/avehtari/BDA_course_Aalto/blob/master/exercises/ex4_rubric.md}{\textbf{here}}

\textbf{Reporting accuracy:} As many significant digits as justified by the Monte Carlo error and posterior accuracy.

To use markmyassignment for this assignment, run the following code in R:
\begin{Schunk}
\begin{Sinput}
> library(markmyassignment)
> exercise_path <- 
+   "https://github.com/avehtari/BDA_course_Aalto/blob/master/exercises/tests/ex4.yml"
> set_assignment(exercise_path)
> # To check your code/functions, just run
> mark_my_assignment()
\end{Sinput}
\end{Schunk}


\HRule

\newpage

\begin{enumerate}

\item {\bf (Bioassay model and importance sampling)}. In this exercise, you will use a dose-response relation model that is used in Section 3.7 of the course book. The used likelihood is the same, but
we will use a different prior distribution on the parameters $\alpha$ and $\beta$.


\begin{itemize}
    \item[a)] Construct a bivariate normal distribution as prior distribution for $(\alpha,\beta)$.
    The marginal distributions are $\alpha \sim N(0,2^2), \beta \sim N(10,10^2)$ with correlation $\mathrm{corr}(\alpha, \beta)=0.5$. Report the mean and covariance
    of the bivariate normal distribution.\\
\textbf{Hint!} The mean and covariance of the bivariate normal distribution are a length-$2$ vector and a $2 \times 2$ matrix. The elements of the covariance matrix can be computed using the relation of correlation and covariance.
  \item[b)] Implement a function in R for computing the \textbf{logarithm} of the density of the
  prior distribution in a) for arbitrary values of $\alpha$ and $\beta$. Below is an example of how the function should be named and work if you want to check them with \texttt{markmyassignment}. 
  
\begin{Schunk}
\begin{Sinput}
> alpha <- 3
> beta <- 9
> p_log_prior(alpha,beta)
\end{Sinput}
\begin{Soutput}
[1] -6.296435
\end{Soutput}
\end{Schunk}
 
\textbf{Hint!} Use function {\tt dmvnorm} from the {\tt aaltobda} package. We use logarithms for better numerical accuracy in later questions.

\item[c)] Implement a function in R for computing the \textbf{logarithm} of the density of the posterior for arbitrary values of $\alpha$ and $\beta$ and data $x$, $y$ and $n$. Below is an example of how the function should be named and work if you want to check them with \texttt{markmyassignment}. 
 
  
\begin{Schunk}
\begin{Sinput}
> library(aaltobda)
> data("bioassay")
> alpha <- 3
> beta <- 9
> p_log_posterior(alpha, beta, x = bioassay$x, y = bioassay$y, n = bioassay$n)
\end{Sinput}
\begin{Soutput}
[1] -15.78798
\end{Soutput}
\end{Schunk}
\textbf{Hint!} Equation (3.16) in the course book. The \textbf{logarithm} of the prior density was already implemented in b). For computing the \textbf{logarithm} of the likelihood, use the {\tt bioassaylp} function from the {\tt aaltobda} package. The data can be loaded with the R command {\tt data("bioassay")}.\\
\textbf{Hint!} Logarithm of the product of two densities is the sum of the log-densities, i.e. $\log ab = \log a + \log b$.

\item[d)] Plot the posterior density in a grid of points ($\alpha \in [-4,4]$, $\beta \in [-10,30]$) using the {\tt bioassay\_posterior\_density\_plot} function from the {\tt aaltobda} package. Internally, it uses the {\tt p\_log\_posterior} function you implemented in c).
\item[e)] Implement two functions in: 1) A function for computing the log importance ratios (log importance weights) for draws from the prior distribution in 1a) when the target distribution is the posterior distribution. 2) A function for exponentiating the log importance ratios and normalizing them to sum to one. Below is a test example, the functions can also be tested with \texttt{markmyassignment}.\\
\begin{Schunk}
\begin{Sinput}
> alpha <- c(1.896, -3.6,  0.374, 0.964, -3.123, -1.581)
> beta <- c(24.76, 20.04, 6.15, 18.65, 8.16, 17.4)
> log_importance_weights(alpha, beta)
\end{Sinput}
\begin{Soutput}
[1]  -8.95 -23.47  -6.02  -8.13 -16.61 -14.57
\end{Soutput}
\begin{Sinput}
> normalized_importance_weights(alpha, beta)
\end{Sinput}
\begin{Soutput}
[1] 0.045 0.000 0.852 0.103 0.000 0.000
\end{Soutput}
\end{Schunk}
\textbf{Hint!} Equation (10.3) in the course book.
\item[f)] Sample draws of $\alpha$ and $\beta$ from the prior distribution from 1a). Implement a function for computing the posterior mean using importance sampling, and report the obtained mean.\\
\textbf{Hint!} Use the function {\tt rmvnorm} from the {\tt aaltobda} package.
\begin{Schunk}
\begin{Sinput}
> posterior_mean(alpha, beta)
\end{Sinput}
\begin{Soutput}
[1] 0.503 8.275
\end{Soutput}
\end{Schunk}
\item[g)] Using the importance ratios, compute the effective sample size $S_{\text{eff}}$ and report it. If $S_{\text{eff}}$ is less than 1000, repeat f) with more draws.\\
\begin{Schunk}
\begin{Sinput}
> S_eff(alpha, beta)
\end{Sinput}
\begin{Soutput}
[1] 1.354
\end{Soutput}
\end{Schunk}
\textbf{Hint!} Equation (10.4) in the course book.
\begin{itemize}
\item \textbf{Note!} {\it BDA3 1st (2013) and 2nd (2014) printing have an error
   for $\tilde{w}(\theta^s)$ used in the effective sample size
   equation (10.4). The normalized weights equation should not have the
   multiplier S (the normalized weights should sum to one). Errata for
   the book
   \url{http://www.stat.columbia.edu/~gelman/book/errata_bda3.txt}.
   The later printings and slides have the correct equation.}
 \end{itemize}
\item[h)] Use importance resampling without replacement to obtain a posterior sample of size 1000 of $\alpha$ and $\beta$ and plot a scatterplot of the obtained posterior sample.
\item[i)] Using the posterior sample obtained via importance resampling, report an estimate for $p(\beta > 0|x, n, y)$, that is, the probability that the drug is harmful.
\item[j)] Using the posterior sample obtained via importance resampling, draw a histogram of the draws from the posterior distribution of the LD50 conditional on $\beta$ > 0.\\
\textbf{Hint!} See Figure 3.4 and corresponding section in the course book. You can plot a basic histogram with R using the library {\tt ggplot2} and the command {\tt ggplot() + geom\_histogram(aes(ld50))}
\end{itemize}



\end{enumerate}



\end{document}

%%% Local Variables:
%%% mode: latex
%%% TeX-master: t
%%% End:
