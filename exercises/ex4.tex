\documentclass[a4paper,11pt]{article}

\usepackage[pdftex]{graphicx}
%\usepackage{babel}
\usepackage[utf8]{inputenc}
\usepackage[T1]{fontenc}
%\usepackage[T1,mtbold,lucidacal,mtplusscr,subscriptcorrection]{mathtime}
\usepackage{times}
\usepackage{amsmath}
\usepackage{url}
\usepackage{enumerate}
\usepackage{parskip}
\usepackage{hyperref}
\usepackage{microtype}


%\usepackage[dvips,bookmarks=false]{hyperref}
% \hypersetup{%
%   bookmarksopen=true,
%   bookmarksnumbered=true,
%   pdftitle={S-114.2601 Bayesilaisen mallintamisen perusteet},
%   pdfsubject={Kommentteja},
%   pdfauthor={Aki Vehtari},
%   pdfkeywords={bayesilainen todennäköisyysteoria, bayesilainen
%     päättely, bayesilaiset mallit, mallien analysointi,
%     laskennalliset menetelmät, Markov-ketju Monte Carlo},
%   pdfstartview={FitH -32768}
% }


% if not draft, smaller printable area makes the paper more readable
\topmargin -4mm
\oddsidemargin 0mm
\textheight 225mm
\textwidth 150mm

%\parskip=\baselineskip

\DeclareMathOperator{\E}{E}
\DeclareMathOperator{\Var}{Var}
\DeclareMathOperator{\var}{var}
\DeclareMathOperator{\Sd}{Sd}
\DeclareMathOperator{\sd}{sd}
\DeclareMathOperator{\Bin}{Bin}
\DeclareMathOperator{\Beta}{Beta}
\DeclareMathOperator{\Poisson}{Poisson}
\DeclareMathOperator{\betacdf}{betacdf}
\DeclareMathOperator{\Invchi2}{Inv-\chi^2}
\DeclareMathOperator{\logit}{logit}
\DeclareMathOperator{\N}{N}
\DeclareMathOperator{\U}{U}
\DeclareMathOperator{\tr}{tr}
\DeclareMathOperator{\trace}{trace}

\pagestyle{empty}

\begin{document}
\thispagestyle{empty}

\section*{Bayesian data analysis -- exercise 4}

This exercise is related to Chapter 3 (additional information on
number of required samples will be covered in Chapter 10).

The maximum amount of points from this assignment is 6. In addition to the correctness of the answers, the overall quality and clearness of the report is evaluated.

Report all results to a single, {\bf anonymous} *.pdf -file and return it to \href{peergrade.io}{peergrade.io}. Include also source code to the report (either as an attachment or as a part of the answer). By anonymity it is meant that the report should not contain your name or student number.

\vspace{1cm}


\subsection*{Generalized linear model: Bioassay + grid sampling}


In the bioassay example (Chapter 3 in the book), replace the uniform prior density by a joint normal prior distribution on $(\alpha, \beta)$, with $\alpha \sim N(0,2^2), \beta \sim N(10,10^2)$, and $\mathrm{corr}(\alpha, \beta)=0.5$.
\begin{enumerate}[a)]
\item Repeat all the computations and plots of Section 3.7 with this
  new prior distribution.
  \begin{itemize}
  	\item Compute the posterior density at a grid of points $(\alpha,\beta)$
  	\item Use the grid to sample 1000 draws from the posterior
  	\item Draw a posterior contour plot for the parameters $\alpha$ and $\beta$ (see Figure 3.3a)
  	\item Draw a scatterplot of the 1000 draws from the posterior (see Figure 3.3b)
  	\item Draw a histogram of the draws from the posterior distribution of the LD50 conditional on $\beta > 0$ (see Figure 3.4)
  \end{itemize}
\item Check visually that your contour plot and scatter plot look like a
  compromise between the prior distribution and the likelihood
\item Report an estimate for $p(\beta>0|x, n, y)$, that is, the probability that the drug is harmful
% Removed because I did not completely understand what is meant here. The new prior doesn't really effectively differ from the uniform prior so
% that conclusions would be that much different.
%\item Discuss the effect of this hypothetical prior information on the
%  conclusions in the applied context.
\end{enumerate}
Hints
\begin{itemize}
\item See {\tt demo3\_6.R}, {\tt demo3\_6.py}, or {\tt demo3\_6.m}
\item Check that the range and spacing of grid for $\alpha$ and
  $\beta$ are sensible for the alternative prior
\item Compute the log-posterior in a grid
\item Scale the log-posterior by subtracting its maximum value before
  exponentiating
\item Exponentiate
\item Normalize the posterior
\item Use 2D grid sampling or the method in the book to sample from the posterior
%\item Multivariate normal log-pdf \texttt{mnorm\_lpdf.m}
%\item Check that shape of the vector x is correct (1xN)
%\item Possibly useful Matlab functions: \texttt{mnorm\_lpdf.m, catrand.m, binsgeq.m, contour}
\end{itemize}





\end{document}

%%% Local Variables:
%%% mode: latex
%%% TeX-master: t
%%% End:
