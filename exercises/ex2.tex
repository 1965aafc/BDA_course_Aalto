\documentclass[a4paper,11pt]{article}

\usepackage[pdftex]{graphicx}
%\usepackage{babel}
\usepackage[utf8]{inputenc}
\usepackage[T1]{fontenc}
%\usepackage[T1,mtbold,lucidacal,mtplusscr,subscriptcorrection]{mathtime}
\usepackage{times}
\usepackage{amsmath}
\usepackage{url}
\urlstyle{same}
\usepackage{enumerate}
\usepackage{parskip}
\usepackage{hyperref}
\usepackage{microtype}

\usepackage{xcolor}
\hypersetup{
    colorlinks,
    linkcolor={red!50!black},
    citecolor={blue!50!black},
    urlcolor={blue!80!black}
}


% if not draft, the smaller printable area makes the paper more readable
\topmargin -4mm
\oddsidemargin 0mm
\textheight 225mm
\textwidth 150mm

%\parskip=\baselineskip

\DeclareMathOperator{\E}{E}
\DeclareMathOperator{\Var}{Var}
\DeclareMathOperator{\var}{var}
\DeclareMathOperator{\Sd}{Sd}
\DeclareMathOperator{\sd}{sd}
\DeclareMathOperator{\Bin}{Bin}
\DeclareMathOperator{\Beta}{Beta}
\DeclareMathOperator{\Poisson}{Poisson}
\DeclareMathOperator{\betacdf}{betacdf}
\DeclareMathOperator{\Invchi2}{Inv-\chi^2}
\DeclareMathOperator{\logit}{logit}
\DeclareMathOperator{\N}{N}
\DeclareMathOperator{\U}{U}
\DeclareMathOperator{\tr}{tr}
\DeclareMathOperator{\trace}{trace}

% Horizontal line
\newcommand{\HRule}{\rule{\linewidth}{0.5mm}}

\pagestyle{empty}

\usepackage{Sweave}
\begin{document}
\input{ex2-concordance}
\thispagestyle{empty}

\section*{Bayesian data analysis -- Assignment 2}

% \HRule


\subsubsection*{General information}

\begin{itemize}
\itemsep0em 
\item The recommended tool in this course is R (with the IDE R-Studio). You can download R \href{https://cran.r-project.org/}{\textbf{here}} and R-Studio \href{https://www.rstudio.com/products/rstudio/download/}{\textbf{here}}. There are tons of tutorials, videos and introductions to R and R-Studio online. You can find some initial hints \href{https://www.rstudio.com/online-learning/}{\textbf{here}}. 
\item  You can write the report with your preferred software, but the outline of the report should follow the instruction in the R markdown template that can be found \href{https://raw.githubusercontent.com/avehtari/BDA_course_Aalto/master/templates/assignment_template.rmd}{\textbf{here}}. 
\item  Report all results in a single, {\bf anonymous} *.pdf -file and return it to \href{peergrade.io}{\textbf{peergrade.io}}. 
\item Many of the exercises can be checked using the R package \texttt{markmyassignment}. Information on how to install and use the package can be found \href{https://cran.r-project.org/web/packages/markmyassignment/vignettes/markmyassignment.html}{\textbf{here}}.
\item The course has its own R package with data and functionality to simplify coding. To install the package just run the following:
\begin{enumerate}
\item \texttt{install.packages("devtools")}
\item \texttt{devtools::install\_github("avehtari/BDA\_course\_Aalto", \\ subdir = "rpackage")}
\end{enumerate}
\item Many of the exercises can be checked automatically using the R package \\ \texttt{markmyassignment}. Information on how to install and use the package can be found \href{https://cran.r-project.org/web/packages/markmyassignment/vignettes/markmyassignment.html}{\textbf{here}}.
\item Additional self study exercises and solutions for each chapter in BDA3 can be found \href{http://www.stat.columbia.edu/~gelman/book/solutions3.pdf}{\textbf{here}}.
\item If you have any suggestions or improvements to the course material, please feel free to create an issue or submit a pull request to the public repository!!
\end{itemize}

\subsubsection*{Information on this assignment}

This exercise is related to Chapters 1 and 2. The maximum amount of points from this assignment is 3. You may find an additional discussion about choosing priors by Andrew Gelman useful, they can be found \href{http://andrewgelman.com/2017/10/04/worry-rigged-priors/}{\textbf{here}}.

\textbf{Reading instructions:} Chapter 1 and 2 in BDA3, see \href{https://github.com/avehtari/BDA_course_Aalto/blob/master/chapter_notes/BDA_notes_ch1.pdf}{\textbf{here}} and \href{https://github.com/avehtari/BDA_course_Aalto/blob/master/chapter_notes/BDA_notes_ch2.pdf}{\textbf{here}}.

\textbf{Grading instructions:} The grading will be done in peergrade. All grading questions and evaluations for assignment 2 can be found \href{https://github.com/avehtari/BDA_course_Aalto/blob/master/exercises/ex1_rubric.md}{\textbf{here}}

To use markmyassignment for this assignment, run the following code in R:
\begin{Schunk}
\begin{Sinput}
> library(markmyassignment)
> exercise_path <- 
+   "https://github.com/avehtari/BDA_course_Aalto/blob/master/exercises/tests/ex2.yml"
> set_assignment(exercise_path)
\end{Sinput}
\end{Schunk}


\HRule

\newpage

\subsection*{Inference for binomial proportion (Computer)}

Algae status is monitored in 274 sites at Finnish lakes and rivers.
The observations for the 2008 algae status at each site are presented
in file {\tt algae.txt} ('0': no algae, '1': algae present). The data
can also be accessed from the {\tt aaltobda} R package as follows:

\begin{Schunk}
\begin{Sinput}
> library(aaltobda)
> data("algae")
> head(algae)
\end{Sinput}
\begin{Soutput}
[1] 0 1 1 0 0 0
\end{Soutput}
\end{Schunk}

% In the year 2008 blue-green algae was observed at 44 sites.
Let $\pi$ be the probability of a monitoring site having detectable
blue-green algae levels.

Use a binomial model for observations and a $\Beta(2,10)$ prior
for $\pi$ in Bayesian inference. Formulate Bayesian model likelihood
$p(y|\pi)$, prior $p(\pi)$, and the resulting posterior $p(\pi|y)$.
Here it is not necessary to derive the posterior distribution as it has already been done in the book.
Also, it is not necessary to write out the distributions; it is sufficient to use label-parameter format, e.g.\ $\Beta(\cdot,\cdot)$. Although recommended, any plotting is not required in this exercise.

As a test case, we provide some results for the following data. It is also possible to check the functions below with \texttt{markmyassignment}.

\begin{Schunk}
\begin{Sinput}
> algea_test <- c(0, 1, 1, 0, 0, 0)
\end{Sinput}
\end{Schunk}

% See exercises/solution folder in teacher repo

Use your model to answer the following questions:
\begin{enumerate}[a)]
\item What can you say about the value of the unknown $\pi$ according
  to the observations and your prior knowledge? Summarize your results
  with a point estimate (i.e. $E(\pi|y)$) and a 90\% interval estimate.
\begin{Schunk}
\begin{Sinput}
> beta_point_est(prior_alpha = 2, prior_beta = 10, data = algea_test)
\end{Sinput}
\begin{Soutput}
[1] 0.2222222
\end{Soutput}
\begin{Sinput}
> beta_interval(prior_alpha = 2, prior_beta = 10, data = algea_test, prob = 0.9)
\end{Sinput}
\begin{Soutput}
[1] 0.06810774 0.43431787
\end{Soutput}
\end{Schunk}
\item What is the probability that the proportion of monitoring sites with detectable algae levels $\pi$ is smaller than $\pi_0=0.2$ that is known from historical records?
\begin{Schunk}
\begin{Sinput}
> beta_low(prior_alpha = 2, prior_beta = 10, data = algea_test, pi_0 = 0.2)
\end{Sinput}
\begin{Soutput}
[1] 0.1382122
\end{Soutput}
\end{Schunk}
\item What assumptions are required in order to use this kind of a
  model with this type of data?
\item Make prior sensitivity analysis by testing a couple of different reasonable priors. Summarize the results by one or two sentences.
\end{enumerate}
\textbf{Hint!} With a conjugate prior, a closed-form posterior is Beta form (see
equations in the book). Useful functions: {\tt dbeta}, {\tt pbeta}, {\tt qbeta} in R.


\end{document}

%%% Local Variables:
%%% mode: latex
%%% TeX-master: t
%%% End:
