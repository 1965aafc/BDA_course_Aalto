
\subsubsection*{General information}

\begin{itemize}
\itemsep0em 
\item The recommended tool in this course is R (with the IDE RStudio). You can download R \href{https://cran.r-project.org/}{\textbf{here}} and RStudio \href{https://www.rstudio.com/products/rstudio/download/}{\textbf{here}}. There are tons of tutorials, videos and introductions to R and RStudio online. You can find some initial hints \href{https://www.rstudio.com/online-learning/}{\textbf{here}}. 
\item  The format of the report should follow the instruction in the R markdown template that can be found \href{https://raw.githubusercontent.com/MansMeg/BDA_course_Aalto/master/templates/assignment_template.rmd}{\textbf{here}}. 
\item  Report all results in a single, {\bf anonymous} *.pdf -file and return it to \href{peergrade.io}{\textbf{peergrade.io}}. 
\item Many of the exercises can be checked using the R package \texttt{markmyassignment}. Information on how to install and use the package can be found \href{https://cran.r-project.org/web/packages/markmyassignment/vignettes/markmyassignment.html}{\textbf{here}}.
\item The course has its own R package with data and functionality to simplify coding. To install the package just run the following:
\begin{enumerate}
\item \texttt{install.packages("devtools")}
\item \texttt{devtools::install\_github("avehtari/BDA\_course\_Aalto", \\ subdir = "rpackage")}
\end{enumerate}
\item Many of the exercises can be checked automatically using the R package \\ \texttt{markmyassignment}. Information on how to install and use the package can be found \href{https://cran.r-project.org/web/packages/markmyassignment/vignettes/markmyassignment.html}{\textbf{here}}.
\item Additional self study exercises and solutions for each chapter in BDA3 can be found \href{http://www.stat.columbia.edu/~gelman/book/solutions3.pdf}{\textbf{here}}.
\item If you have any suggestions or improvements to the course material, please feel free to submit a Pull Request to the public repository!
\end{itemize}